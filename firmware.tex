\chapter{MUEB firmware}

\section{Tábla kommunikációs protokoll}
\section{Ethernet protokoll}
Két porton hallgat a MUEB, az egyiken parancsokat lehet neki adni, a másikon pedig animációkat fogad.
\par
Utasítás port: 3000
\par
Animáció port: 2000

\subsection{Parancsok}
\begin{enumerate}
  \item 12V lekapcsolása <-- mátrix végén a csatlakozók nem lesznek feszültség alatt a leszereléskor
  \item 12V bekapcsolása (a véletlen kikapcsolások esetére)
  \item Reboot
  \item Status lekérdezés
  \item Beépített animáció indítása
  \item Beépített animáció leállítása, ekkor az animációk portjára érkezett csomagok jelennek meg.
\end{enumerate}
\subsection{Parancs csomagok felépítése}
A parancs első három bájtja az 'S', 'E' illetve 'M' karakterek ebben a sorrendben, az ezt követő első bájt az utasítás, majd ezt követik az utasítások esetleges argumentumai. 
\subsection{Animáció csomagok felépítése}
Az első bájt szabja meg, hogy melyik ablaknak, és melyik pixelnek szól a csomag. A bájt MSB-je mondja meg, hogy melyik ablak. Ha a bit be van állítva, akkor az kívülről nézve jobb oldali ablak, ha nincs beállítva, akkor kívülről nézve bal ablak. A bájt alsó bitjei határozzák meg a pixelt. Az alsó két bithatározza meg a pixelt. Kívülről nézve bal felső a 0, jobb felső az 1, bal alsó a 2, és jobb alsó a 3. Az első bájt nem használt bitjei 0-ra legyenek állítva. 
\par
A második bájt ábrázojl a pixel vörös komponensét, a harmadik bájt a zöld komponensét, a negyedik bájt a kéket. A skálán a 255 a legfényesebb, a 0 a teljesen kikapcsolt.
\par
A csomagban nem szereplő pixelek változatlan állpotban maradnak.
\subsection{MAC cím}
\section{LED-ek}

\section{Ablak automata}
%\begin{tikzpicture}[shorten >=1pt,node distance=2cm,on grid,auto] 
\begin{tikzpicture}[>=stealth',shorten >=2pt,auto,shorten <=2pt,node distance=5 cm,
                    every state/.style={align=center,minimum size=3cm,text width=2cm},
		    every edge/.style={draw,thick},
		    loop label/.style={draw,align=center,text width=2cm,outer sep=4pt,minimum height=1cm}
		   ]

   \node[state,initial] (vcc_3v3_off)   {3.3V táp kikapcsolva}; 
   \node[state] (vcc_3v3_on) [right=of vcc_3v3_off] {3.3V táp bekapcsolva, de még nincs sikeres kommunikáció}; 
   \node[state] (vcc_12v_off) [below=of vcc_3v3_on] {12V-os táp kikapcsolva}; 
   \node[state,accepting](vcc_12v_on) [left=of vcc_12v_off] {12V-os táp bekapcsolva};
   \path[->] 
          (vcc_3v3_off) edge  node {} (vcc_3v3_on)
          (vcc_3v3_on) edge  node  {üdvözlő üzenet fogadva} (vcc_12v_on)
	  (vcc_3v3_on) edge  node  {Timeout} (vcc_3v3_off)
          (vcc_12v_on) edge  node  {Parancs} (vcc_12v_off)
          (vcc_12v_off) edge  node  {Parancs} (vcc_12v_on);
\end{tikzpicture}
\par
A MUEB 5 másodperc után bekapcsolja a PWM panel tápellátását, majd vár egy másodpercet az üdvözlőüzenet fogadására. Amennyiben ez nem történik meg, kikapcsolja a csatlakozón a 3.3V-os tápfeszültséget, és 5 másodperc után újra próbálkozik.
\par
Sikeres kommunikációt követően a MUEB bekapcsolja a 12V-ot a táblára. A 12V-ot hálózaton keresztül kapcsolhatjuk ki a megfelelő paranccsal, illetve ebből az állapotból vissza.
\par
Az állapotgép újraindítható szintén egy hálózat felől érkező paranccsal, ekkor a tápelltás megszűnése miatt a kijelző panel is újraindul.
