\chapter{MUEB firmware}

\section{Tábla kommunikációs protokoll}
\section{Ethernet protokoll}
Két porton hallgat a MUEB, az egyiken parancsokat lehet neki adni, a másikon pedig animációkat fogad.
\par
Utasítás port: 3000
\par
Animáció port: 2000

\subsection{Parancsok}
\begin{enumerate}
  \item 12V lekapcsolása <-- mátrix végén a csatlakozók nem lesznek feszültség alatt a leszereléskor
  \item 12V bekapcsolása (a véletlen kikapcsolások esetére)
  \item Reboot
  \item Status lekérdezés
  \item Beépített animáció indítása
  \item Beépített animáció leállítása, ekkor az animációk portjára érkezett csomagok jelennek meg.
\end{enumerate}
\subsection{Parancs csomagok felépítése}
A parancs első három bájtja az 'S', 'E' illetve 'M' karakterek ebben a sorrendben, az ezt követő első bájt az utasítás, majd ezt követik az utasítások esetleges argumentumai. 
\subsection{Animáció csomagok felépítése}
Az első bájt szabja meg, hogy melyik ablaknak, és melyik pixelnek szól a csomag. A bájt MSB-je mondja meg, hogy melyik ablak. Ha a bit be van állítva, akkor az kívülről nézve jobb oldali ablak, ha nincs beállítva, akkor kívülről nézve bal ablak. A bájt alsó bitjei határozzák meg a pixelt. Az alsó két bithatározza meg a pixelt. Kívülről nézve bal felső a 0, jobb felső az 1, bal alsó a 2, és jobb alsó a 3. Az első bájt nem használt bitjei 0-ra legyenek állítva. 
\par
A második bájt ábrázojl a pixel vörös komponensét, a harmadik bájt a zöld komponensét, a negyedik bájt a kéket. A skálán a 255 a legfényesebb, a 0 a teljesen kikapcsolt.
\par
A csomagban nem szereplő pixelek változatlan állpotban maradnak.
\subsection{MAC cím}
\section{LED-ek}
