\chapter{Hardver}

\section{Visszajelző szervek}
A NYÁK-on 8 db LED található.
A szoftverből vezérelt LED-ek funkciójáról részletesebben a %TODO
fejezetben olvashatnak.

\subsection{12V feliratú LED}
A 12V feliratú LED jelzi, hogy a csatlakoztatott külső tápegység feszültséget ad ki magából, illetve ép az áramkör bemenetén lévő 1,5A-es biztosíték.
\subsection{3.3V feliratú LED}
A 3.3V feliratú LED jelzi, hogy a %TODO
%TODO ablak ledek

\subsection{Heartbeat LED}
A mikrokontroller épségét jelzi vissza. A szoftver villogtatja másodperc nagyságú periódusidővel.
\subsection{DHCP LED}
A DHCP-vel kapcsolatos 

\section{ESD}
Az eszközt ESD pisztollyal teszteltük a HVT-n. \par
8kV-al mellette elsütve a pisztolyt nem tapasztaltunk semmit.
\par
4kV közevetlen a NYÁK-ba sütve a pisztolyt legrosszabb esetben a visszajelző LED kiégett vagy leállt az animáció. Újraindítás után működött tovább az eszköz.
A panel csatlakozó és az SWD interfészen nem tapasztaltunk semmit.
A földre sütve a pisztolyt legrosszabb esetben lefagyott a mikrokontroller, újraindítás után már működött.
\par
8kV-on tesztelve a 4kV-oshoz hasonló eredményt kaptunk.

