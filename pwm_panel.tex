\chapter{A kijelzők vezérlőpanelje}

\section{Koncepció}
Egy kijelző 4 darab pixelt tartalmaz. A panel egy szalagkábelen keresztül csatlakozik a MUEB-hez, azzal UART-on kommunikál.
A két ablakot egy jumperelés konfigurálja "ellentétes" funkcióra.

\subsection{Pixelek}
Minden pixel 9 darab alpixelből áll össze. Az alpixeleken belül három azonos színű LED van sorba kötve, majd ezek a blokkok párhuzamosan.
%todo a táblán lévő ledek elhelyezkedése
%todo szalagkábelcsati kiosztása
%todo elosztó panel dokumentálása

\section{Technikai specifikáció} %todo táblázat
Baud rate: %todo
\par
V+: 12V
%todo áramfelvétel
\par
Vcc: 3.3V
%todo áramfelvétel

\section{Kommunikációs protokoll}
\begin{center}
  \begin{tabularx}{\linewidth}{ L L }
    Funkció & Küldendő byte \\ \hline \hline
    OldalA panel kiválasztása & F0 \\ \hline
    OldalB panel kiválasztása & F1 \\ \hline
    Verziószám visszaküldése \par A felső hét bit a verzió, az alsó bit az ablak jumperelése & 1101xxxx \\ \hline
    Mindkét ablak fázistolásának párhuzamos \par írása alpixelekre vonatkoztatva burstösen & 1110xxxx, ahol xxxx a fázistolás mértéke  \\ \hline
    Alpixel fényerejének állítása a kiválasztott ablakon & xxxxyyyy, ahol xxxx az alpixel sorszáma, yyyy a fényerő \\ \hline
  \end{tabularx}
\end{center}

Javaslatok:
\begin{enumerate}
 \item Forráskódolás a kódszókon
 \item 8 bites mélység
\end{enumerate}

\section{Alpixelek számozása}
%todo
